\begin{enumerate}[label=\thesubsection.\arabic*.,ref=\thesubsection.\theenumi]
\numberwithin{equation}{enumi}

\item
The non-inverting op-amp configuration shown in fig.\ref{fig:original_circuit} provides direct implementation of feedback loop.Assuming operational amplifier has infinite input resistance and zero output resitance.Find the expression for feedback factor.
\begin{figure}[!ht]
	\begin{center}
		
		\resizebox{\columnwidth}{!}{\begin{circuitikz}
\ctikzset{bipoles/length=1cm}

\draw 
(0, 0) node[op amp] (opamp) {}
(opamp.-) -- (-1.5,0.35) to[R,l_=$R_1$,*-*] (-3, 0.35) to (-3.5, 0.35) to (-4, 0.35) node[ground]{}
(opamp.-) --(-0.9,1) to[R=$R_2$] (1,1) -- (1,0) --(2,0) node at(2.3,0){$V_0$}
(opamp.out) to (1.5,0)--(1.5,-0.5) to[R=$R_L$] (1.5,-1.5) to (1.5,-1.5) node[ground]{}
(opamp.+) -- (-0.6,-0.35) to[R =$R_s$,*-*] (-2.6,-0.35) to[V=$V_s$] (-2.6,-2.4) node[ground]{}
node at(-3.25,0.1){$-$}
node at(-1.5,0.1){$+$}
node at(-2.4,0){$V_f$}
;\end{circuitikz}

}
	\end{center}
\caption{}
\label{fig:original_circuit}
\end{figure}
\solution Let the gain of the operational amplifier be G.
The equivalent circuit of the amplifier is in fig.\ref{fig:equivalent_circuit}
\begin{figure}[!ht]
	\begin{center}
		
		\resizebox{\columnwidth}{!}{\usetikzlibrary{decorations.markings}
\begin{circuitikz}
\ctikzset{bipoles/length=1cm}

\draw 
(0, 0) to[V=$V_s$] (0,-1.5) to (0,-1.5) node[ground]{}
(0,0) -- (0,1)--(0.25,1) to[R=$R_s$] (1.5,1)  node at(1.8,1){$+$}
%(1.5,3) node[pos=10]{$V_i$}
(1.5,-1.25)  node at(1.7,-1.25){$-$} 
(1.5,-1.25) -- (1,-1.25) -- (1,-1.75) to[R=$R_1$] (1,-2.75) --(1,-3) node[ground]{}
(1,-1.5) to[R=$R_2$] (5,-1.5){}
(5,-1.5) -- (5,1) --(3.5,1) to[V=$GV_i$] (3.5,-0.5) node[ground]{}
(5,1) --(6,1) to[R=$R_l$,*-*] (6,-0.5) node[ground]{}
(6,1) --(6.5,1) node at(6.8,1){$V_0$}
node at(1.8,-0.3) {$V_i$}
;\end{circuitikz}
}
	\end{center}
\caption{}
\label{fig:equivalent_circuit}
\end{figure}
From the equivalent circuit,
Applying Ohms law,
\begin{align}
V_0 &= G(V_+ - V_-) \label{eq:opamp_output}
\end{align}
Now,Applying voltage dividing rule
\begin{align}
V_- &= \sbrak{\frac{R_1}{R_1+R_2}}V_0
\end{align}
Substituting in equ.\ref{eq:opamp_output}
\begin{align}
    V_0 &= G(V_+-\sbrak{\frac{R_1}{R_1+R_2}}V_0)
    \\
\implies V_0 &= GV_+-G\sbrak{\frac{R_1}{R_1+R_2}}V_0
    \\
G(V_+)&=V_0+G\sbrak{\frac{R_1}{R_1+R_2}}V_0
\end{align}
But,
\begin{align}
    V_s &= V_+
\end{align}
because, no current flows through resistor,Rs Since,input resistance is given infinite in fig.\ref{fig:equivalent_circuit} 
The equation can be written as...,
\begin{align}
    V_0 &= G\sbrak{\frac{1}{1+\frac{GR_1}{R_1+R_2}}}V_s
    \\
 \text{Gain =}\frac{V_0}{V_s}&=\sbrak{\frac{G}{1+\frac{GR_1}{R_1+R_2}}}\label{eq:opamp_gain}
\end{align}
For a negative feedback system,
\begin{align}
   \frac{V_0}{V_i} &= \frac{G}{1+GH}
  \\
  \text{where.,} H = \frac{R_1}{R_1+R_2}
\end{align}
The equation.\ref{eq:opamp_output} looks exactly similar to the Gain of a negative feedback system with
\begin{itemize}
    \item Open loop gain = G
    \item Loop gain = P 
    \item Amount of feedback = F
    \item Feedback factor = f
    \item closed loop gain = T
\end{itemize}
where
\begin{align}
    G &= G\\
    P &= GH = G\frac{R_1}{R_1+R_2}\\
    F &= 1+GH= 1 + \frac{GR_1}{R_1+R_2}\\
    f &= H = \frac{R_1}{R_1+R_2}
\end{align}
Therefore,This operational amplifier can be modelled as a negative feedback system shown in the fig.\ref{fig:equivalent_control_system}
\begin{figure}[!ht]
	\begin{center}
			\resizebox{\columnwidth}{!}{
\tikzstyle{block} = [draw, fill=blue!20, rectangle, 
    minimum height=3em, minimum width=4em]
\tikzstyle{sum} = [draw, fill=blue!20, circle, node distance=1cm]
\tikzstyle{input} = [coordinate]
\tikzstyle{output} = [coordinate]
\tikzstyle{pinstyle} = [pin edge={to-,thin,black}]

\begin{tikzpicture}[auto, node distance=2cm]
    \node [input, name=input] {$V_s$};
    \node [sum, right of=input] (sum) {};
    \node [block, right of=sum] (controller) {$G(s)$};
    \node [output, right of=controller] (output) {};
    \node [block, below of=controller] (feedback) {$H = 0.099$};
    \draw [draw,->] (input) -- node {$V_s$} (sum);
    \draw [->] (sum) -- node {$V_i$} (controller);
    \draw [->] (controller) -- node [name=y] {$V_o$}(output);
    \draw [->] (y) |- (feedback);
    \draw [->] (feedback) -| node[pos=0.99]{$-$}  node [near end] {$V_f$} (sum);
\end{tikzpicture}
}
	\end{center}
\caption{}
\label{fig:equivalent_control_system}
\end{figure}
So, the feedback factor f..,
\begin{align}
     f &= H = \frac{R_1}{R_1+R_2}
\end{align}
\item Find the condition under which closed loop gain T is almost entirely determined by the feedback network.
\solution For T to entirely dependent on feedback network, it should be independent on G(open loop gain)
T is given by..,
\begin{align}
    T &= \frac{G}{1+\frac{GR_1}{R_1+R_2}} \\
\end{align}
For T to be independent on G..,
\begin{align}
 GH >> 1 \\
 G\frac{R_1}{R_1+R_2} >> 1 \\
 G >> 1 + \frac{R_2}{R_1} 
\end{align}
Under such condition..,
\begin{align}
    T &= \frac{1}{H} \\
    T &= \frac{R_1+R_2}{R_1}\\
    T &= 1+\frac{R_2}{R_1}
\end{align}
so, the necessary condition for Af depend only on feedback network is
\begin{align}
    G >> T
\end{align}
\item If the open loop voltage gain
\begin{align} 
G & = 10^4
\end{align}
Find the ratio of R2 and R1 to obtain a closed loop gain of 10.
\solution The closed loop gain gain T is given by
\begin{align}
    T &= \frac{G}{1+GH}
        = \frac{G}{1+\frac{GR_1}{R_1+R_2}} = 10\\
    \text{where..,} G &= 10^4 \\
    10 &= \frac{10^4}{1+\frac{10^4}{1+\frac{R_2}{R_1}}}\\
\implies 1+\frac{R_2}{R_1} &= \frac{10^4}{\frac{10^4}{10}-1}
\\
1+\frac{R_2}{R_1} &= 10.010
\\
\frac{R_2}{R_1} &= 9.010
\end{align}
\item What is the amount of feedback in decibels?
\solution The value of F in decibals is given by 
\begin{align}
    F(dB) &= 20log(F)\\
\text{where..,} F &= 1+GH \\
F &= \frac{G}{T}\\
\text{where..,} G&=10^4 \\ T &= 10\\
F(dB) &= 20log(\frac{10^4}{10})=20log(1000)\\
F(dB) &= 60 dB
\end{align}
\item If G decreases by 20\%,what is the corresponding decrease in T?
\solution Given
\begin{align}
G = 10^4
\end{align}
If G decrease by 20\% then,
the value of G is..,
\begin{align}
    G &= (1-0.2)10^4 \\
      &= 8000
\end{align}
For this value of G and ,
\begin{align}
    \frac{R_2}{R_1} = 9.010
\end{align}
The value of T can be solved as follows,
\begin{align}
 T &= \frac{G}{1+\frac{G}{1+\frac{R_2}{R_1}}}\\
 T &= \frac{8000}{1+\frac{8000}{1+0.9010}}\\
 T &= 9.99749
\end{align}
The percentage change in T is..,
\begin{align}
    fractionalchange &= \frac{10-9.99749}{10}\\
      &= 2.51x10^{-4}\\
     \% change in T &= 0.00251
\end{align}
Therefore T decreases by 0.0025\% when G decreases by 20\%
\end{enumerate}

